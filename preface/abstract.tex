% Special indentation for abstract.
\setlength{\parskip}{1em}
\setlength{\parindent}{0em}

\noindent
Robots are awesome. The word robot was introduced to the public by the Czech
interwar writer Karel Čapek in his play R.U.R. (Rossum's Universal Robots),
published in 1920. The play begins in a factory that uses a chemical substitute
for protoplasm to manufacture living, simplified people called robots. The play
does not focus in detail on the technology behind the creation of these living
creatures, but in their appearance they prefigure modern ideas of androids,
creatures who can be mistaken for humans. These mass-produced workers are
depicted as efficient but emotionless, incapable of original thinking and
indifferent to self-preservation. At issue is whether the robots are being
exploited and the consequences of human dependence upon commodified labour
(especially after a number of specially-formulated robots achieve
self-awareness and incite robots all around the world to rise up against the
humans).
